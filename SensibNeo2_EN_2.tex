% !TEX TS-program = pdflatex
% !TEX encoding = UTF-8 Unicode

% This is a simple template for a LaTeX document using the "article" class.
% See "book", "report", "letter" for other types of document.

\documentclass[11pt]{article} % use larger type; default would be 10pt

\usepackage[utf8]{inputenc} % set input encoding (not needed with XeLaTeX)
\usepackage[french]{babel}

%%% Examples of Article customizations
% These packages are optional, depending whether you want the features they
%provide.
% See the LaTeX Companion or other references for full information.

%%% PAGE DIMENSIONS
\usepackage{geometry} % to change the page dimensions
\geometry{a4paper} % or letterpaper (US) or a5paper or....
% \geometry{margin=2in} % for example, change the margins to 2 inches all round
% \geometry{landscape} % set up the page for landscape
%   read geometry.pdf for detailed page layout information

\usepackage{graphicx} % support the \includegraphics command and options

% \usepackage[parfill]{parskip} % Activate to begin paragraphs with an empty
%line rather than an indent

%%% PACKAGES
\usepackage{booktabs} % for much better looking tables
\usepackage{array} % for better arrays (eg matrices) in maths
\usepackage{paralist} % very flexible & customisable lists (eg.
%enumerate/itemize, etc.)
\usepackage{verbatim} % adds environment for commenting out blocks of text & for
%better verbatim
\usepackage{subfig} % make it possible to include more than one captioned
%figure/table in a single float
\usepackage{longtable}
\usepackage{tabularx}
% These packages are all incorporated in the memoir class to one degree or
%another...

%%% HEADERS & FOOTERS
\usepackage{fancyhdr} % This should be set AFTER setting up the page geometry
\pagestyle{fancy} % options: empty , plain , fancy
\renewcommand{\headrulewidth}{0pt} % customise the layout...
\lhead{}\chead{}\rhead{Training Method : Awareness to Computer Security for a Neophyte Audience -
Tiphaine Romand-Latapie}
\lfoot{\thepage}\cfoot{}\rfoot{\tiny{Copyright Orange \& Tiphaine Romand-Latapie\\
Any reproduction, representation, use or modification 
without the consent of the author  is prohibited.\\
Orange – SA au capital de 10 595 541 532 e - 78 rue Olivier de Serres - 75505
Paris Cedex 15 - 380 129 866 RCS Paris \\
Tiphaine Romand-Latapie : tiphaine.romand@orange.com}
}

%%% SECTION TITLE APPEARANCE
\usepackage{sectsty}
\allsectionsfont{\sffamily\mdseries\upshape} % (See the fntguide.pdf for font
%help)
% (This matches ConTeXt defaults)

%%% ToC (table of contents) APPEARANCE
\usepackage[nottoc,notlof,notlot]{tocbibind} % Put the bibliography in the ToC
\usepackage[titles,subfigure]{tocloft} % Alter the style of the Table of
%Contents
\renewcommand{\cftsecfont}{\rmfamily\mdseries\upshape}
\renewcommand{\cftsecpagefont}{\rmfamily\mdseries\upshape} % No bold!

%%% END Article customizations

%%% The "real" document content comes below...

\title{Training Method : Awareness to Computer Security for a Neophyte Audience v0.2}
\author{Tiphaine Romand-Latapie} 
\date{06/05/2015} % Activate to display a given date or no date (if empty),
         % otherwise the current date is printed 

\begin{document}
\maketitle

\begin{abstract}
%Ce document décrit une méthode de sensibilisation d'un public néophyte à la
%sécurité informatique. Cette méthode est basée sur l'utilisation d'un jeu de
%rôle, inventé par l'auteur. Le lecteur trouvera dans ce document les
%informations lui permettant de réaliser lui-même cette formation (soumis à
%autorisation de l'auteur).
The document describes how to train a neophyte audience to the
basic principles of Computer Security. This method is based on a
role playing game, invented by the author. The reader will find in
this document the information needed to carry out the training.
\end{abstract}

\section{Copyright}
%Ce document, la formation qu'il décrit et la méthodologie présentée sont
%propriétés de leur auteur, Tiphaine Romand-Latapie. La formation a été réalisée
%pour la première fois le 06 Mai 2015, à partir d'une idée originale de l'auteur.
%Le document est protégé par un Copyright Orange \& Tiphaine Romand-Latapie.
%Toute reproduction, représentation, utilisation ou modification est interdite
%sans autorisation de l'auteur.
This document, the training and the methodology presented are
properties of the author. The training has been carried out for the
first time on May 6th 2015, based on a original idea of the
author. This document is protected by a CopyRight Orange \&
Tiphaine Romand-Latapie. Any reproduction, representation, use or
modification without the consent of the author is prohibited.
\\
%Joindre l'auteur : 
Contact: 
\begin{verbatim}tiphaine.romand@orange.com\end{verbatim}
\begin{verbatim}tiphaineRL@gmail.com\end{verbatim}

%\section{Introduction}
%L'idée de cette formation est née de la nécessité de former un public
%opérationnel néophyte aux enjeux de la sécurité informatique. Plutôt que de
%rentrer dans un mécanisme de formation standard, basé sur la compréhension du
%contexte technique (qu'est-ce qu'un mot de passe, comment fonctionne un
%ordinateur etc.),
%qui, selon mon expérience, a tendance à ennuyer ou effrayer un public non
%informaticien, j'ai souhaité me concentrer sur les principes génériques de la
%sécurité informatique:
The concept of this methodology is born from the need to train an 
operational and neophyte audience to the computer security 
stakes. According to the author experience, standard training
 focused on the technical context (what is a password is, how 
 does a computer work etc.) tends to bore or scare a neophyte audience. 
 An alternative would rather be to concentrate on the generic principles of InfoSec:
\begin{itemize}
%\item La problématique de « faire confiance » à une entité/personne ;
%\item La notion de « défense en profondeur » ;
%\item	Les motivations de l'attaquant ;
%\item La démystification de l'attaquant (qui n'est pas forcément un « hacker de
%génie ») ;
%\item	Les notions de compromis entre contraintes opérationnelles et sécurité  ;
%\item Les objectifs des équipes de sécurité (prévoir le comportement de
%l'attaquant, le prévenir ou le détecter) .
\item The decision whether or not to trust an entity/a person 
\item The notion of in-depth defense  
\item The attacker's motivations
\item The attacker : stereotype versus reality (he is not necessarily a
 ``genius hacker'')
\item The necessary trade-off between operationnal constraint and security
\item The goals of the security team (forecast the attacker's behaviour, 
prevent or detect the attack). 
\end{itemize}
%L'idée sur laquelle est basée cette formation est la suivante : les macro
%principes suivants sont les mêmes en sécurité physique et informatique :
The concept of the training stems from the fact that
basics principles, in particular the followings,
 are the same for physical or computer security :
\begin{itemize}
%\item Nous sommes tous les jours confrontés à la notion de « faire confiance » à
%quelqu'un
%\item En sécurité physique, nous travaillons toujours dans le pire cas, nous
%traitons de plus les cas où la mesure de sécurité préliminaire est
%désactivée/inopérante
%\item	L'attaquant est motivé par l'argent/l'idéologie etc.
%\item	Les attaquants les plus répandus ne sont pas des génies du crime 
\item In every day life, we have to decide who we trust.
\item In physical security, we always work on the worst-case scenario and we 
handle the cases where the basic security measure is deactivated/ineffective. 
\item Attackers' motivation are money, ideology etc. 
\item The most common attacker is a not crime genius.
\end{itemize}
%La sécurité physique impose elle aussi des contraintes (fermer la porte à clé,
%port de badge obligatoire, contrôle avant de prendre l'avion, etc.)
%Les objectifs sont ici communs : prévoir le comportement d'un attaquant, le
%prévenir ou le détecter.
%
%Or, les personnes néophytes sont beaucoup plus familières avec la sécurité
%physique que la sécurité informatique, autant dans leur vie professionnelle que
%personnelle. Tout le monde ferme sa porte à clé avant de sortir de chez soi,
%personne ne laisse rentrer n'importe qui chez lui, tout le monde a déjà eu à
%faire à des contrôles de sécurité, etc.
%
%L'idée au cœur de la formation est donc de faire prendre conscience au public
%formé qu'il dispose déjà des bons réflexes et raisonnements, et de lui apprendre
%à les appliquer à la sécurité informatique, tout en dédramatisant cette
%dernière.

Physical security brings constraints too (lock the door, 
carry a badge, perform security check at the airport etc.). 
These constraints serve the same goal: forecast the attacker's behaviour, 
prevent it or detect it.

Yet, a neophyte audience is more familiar with physical security 
than with computer security, in their everyday life or professional life. We 
lock the door before going out, we don't let anybody enter 
our home, we've all already gone through security checks etc.

The core idea of this training is therefore to make neophyte people realize 
that they already know security best practices. They only have to learn how 
to apply them to computer security: they do so in a fun way, while playing 
the game. 

%\section{Le jeu de rôle}
%La formation est construite autour d'un jeu de rôle basé sur l'attaque et la
%défense d'un bâtiment.

\section{The role playing game}
The training is developped around a role playing game consisting in 
attacking and defending a building.

%\subsection{Règles du Jeu}
%Le jeu se déroule avec un Animateur, appelé aussi « Maître du Jeu », une équipe
%d'attaquants et une équipe de défenseurs.

\subsection{Rules}
The game is led by a Game Master (GM) and involves an attack team and a defense team.

%\subsubsection{Description générale :}
\subsection{General description}

\begin{itemize} 
%\item Un immeuble de bureau dans une zone dense, avec parking souterrain et
%hélipad (piste d'atterrissage d'hélicoptère sur le toit). Un objet (tenant dans
%un sac à dos) de grande valeur, utilisé par des employés pendant la journée est
%stocké quelque part dans l'immeuble.
% \item Au début du jeu l'immeuble n'est pas sécurisé. 
%\item Les attaquants proposent une attaque, les défenseurs une contre-mesure, et
%on recommence. 
\item The action takes place in an office building located in a dense urban area, 
with an underground parking lot and an helicopter landing strip. A
highly valuable object (fitting in a backpack), used by
employees during the day, is stored somewhere in the building.
\item At the beginning of the game, the building is not secured
at all.
\item The attackers propose an attack, the defenders a
mitigation, in an iterative way.
\end{itemize}

%\subsubsection{Règles et objectifs de l'équipe d'attaquants :}
\subsubsection{Attack team's rules and goals}
\begin{itemize}
 %\item Objectif : voler l'objet dans l'immeuble sans finir en prison
%\item Règles : budget illimité – nombre d'attaquants humains inférieur à dix -
%respecter les lois de la physique
\item Goals: steal the object without being caught.
\item Rules: unlimited budget, limited number of human attackers 
in the game (no more than ten person). Physics rules apply (gravity, etc).
\end{itemize}

%\subsubsection{Règles et objectifs de l'équipe de défenseurs :}
\subsubsection{Defense team's rules and goals}
\begin{itemize}
%\item Objectif : empêcher le vol de l'objet ou récupérer de quoi faire arrêter
%les voleurs
%\item Règles : Budget illimité – personnel illimité – respecter les lois
%françaises et les lois de la physique – des gens doivent pouvoir travailler dans
%l'immeuble pendant la journée
\item Goals: prevent the theft or retrieve data allowing to catch the attackers.
\item Rules: unlimited budget, unlimited staff. Physics rules apply, the law must be respected, 
employees must be able to work in the building during office hour.
\end{itemize}


%\subsubsection{Fin d'un scenario :}
\subsubsection{End of a scenario}
\begin{itemize}
%\item Je conseille de terminer un échange (appelé « scenario ») lorsque les
%équipes arrivent à un point de blocage (tout le monde est mort, l'objet est
%détruit, les policiers sont là …)
%\item Il est alors possible de passer à une nouvelle tentative (reprise à zéro)
%des attaquants. Dans ce cas : les défenseurs conservent toutes leurs mesures de
%protection.
%\item Dans le cas où les joueurs le souhaitent, ou si le maître du jeu souhaite
%relancer le jeu, il est également possible d'inverser les équipes (les
%attaquants deviennent défenseurs etc.)
\item I recommend to stop the current exchange (called ``scenario'' in the following) 
when teams get to a blocking point (everybody is dead, the object is destroyed, 
the police has arrived ...). 
\item It is then possible to move on a new try of the attack team. 
The defenders are keeping all the security measure already deployed. 
\item If the players want to, or if the GM wants to revive the game, it is possible to switch the team: 
the attackers become the defenders and vice versa.
\end{itemize}



%\subsubsection{Fin du jeu :}
\subsubsection{End of the game}
\begin{itemize}
%\item Il n'y a pas de gagnant ni de perdant ! 
%\item Il est conseillé de faire plusieurs échanges ou « scenarii » dans une
%seule session, en ce cas, la fin du jeu est laissée à la discrétion de
%l'animateur (nous conseillons un jeu d'une durée de 40 à 60 minutes pour 6
%personnes).
%\item Le jeu est suivi d'un débriefing par le formateur, permettant de mettre en
%exergue les notions souhaitées (voir la section consacrée au débriefing).
\item There is neither winner nor loser!
\item I recommend to do multiple scenarii during one game. The duration of the session
 is a choice of the GM, (forty to sixty minutes is a good duration for a 6 player game).
\item The session is followed by a debriefing by the trainer, allowing him or her 
to highlight the concepts (see the ``Debriefing'' section).
\end{itemize}
	
%\subsection{Concept des règles}
\subsection{Behind the rules}
%\subsubsection{Environnement}
\subsubsection{The playing environment}
%L'environnement du jeu (immeuble de bureau, zone dense, etc.) a été choisi pour
%maximiser le côté ludique et faciliter l'application à la formation :
The playing environment (building, dense area etc.) was chosen to maximize 
the playful side of the game and facilitate its application to the training: 

%%% Fin Juju 

\begin{itemize}
%\item L'immeuble de bureau utilisable pendant la journée permet de travailler
%sur les compromis contrainte/sécurité et offre un environnement familier des
%joueurs.
\item The fact that the building must be usable by employee during the day allows the trainer to work on security versus constraint compromises and offer a familiar environment for the players.
	\begin{itemize}	
%	\item Ne pas hésiter à personnaliser les détails du scénario avec
%	l'environnement professionnel des joueurs : bâtiment de la société, objet de
%	valeur correspondant à un produit phare de l'entreprise, etc. Cela permet une
%	immersion et implication des joueurs plus rapide (sans compter le plaisir à
%	virtuellement perturber le quotidien professionnel).
	\item It can be a good idea to personalize the details of the game using the players' professional 
	environment : company's building, its key product, etc. This allows a faster immersion
	and involvement from the players.
	\end{itemize}
%\item Le choix de la zone dense, de l'hélipad et du parking souterrain permet de
%renforcer le côté ludique (les attaquants peuvent imaginer creuser un tunnel, se
%poser en hélicoptère, sauter d'un immeuble à l'autre etc.) et de guider un peu
%les joueurs. Cela permet également de forcer la diversification des scenarii
%d'intrusion.
%\item Le choix de ne pas plus préciser l'environnement permet de laisser libre
%cours à l'imagination des joueurs, et de simplifier les règles.
%\item L'utilisation de l'objet pendant la journée permet d'éviter des mesures
%non constructives pour le jeu, du type « je coule l'objet dans un bloc de béton ».
%\item L'emplacement de l'objet est laissé libre, il peut évoluer au cours du jeu.
\item The choice of a dense area, as well as the helicopter landing strip 
and the underground parking lot, reinforce the fun part 
(the attackers can think of helicopter landing on the roof, 
can jump from a building to another etc.) 
and guides the players. Furthermore, it helps diversify the scenarii. 
\item The choice not to further detail the environment has been 
made to let the players' imagination run wild and to simplify the rules of the game.
\item The usability of the object during office hour allow us to stay clear 
of non constructive mitigation, like ``we cast the object in concrete''.
\item The location of the object within the building is let free, 
it can change during the game if the defenders wish so.
%\item	Le fait de commencer sans sécurité est important : 
\item Beginning with a non-secured building is important :
	\begin{itemize}	
	%\item Il permet de travailler sur l'empilement des mesures de sécurité et sur le
	%principe selon lequel l'attaquant passe toujours par le point de moindre
	%résistance ;
	%\item Les attaquants partent régulièrement du principe que l'immeuble « sans
	%sécurité » comporte quand même des caméras de surveillance, des portes qui
	%ferment à clé etc. Dans ce cas il n'est pas nécessaire de recadrer le jeu. En
	%revanche, il est intéressant de faire réfléchir les joueurs sur ce sujet au
	%cours du débriefing.
	\item It allows the trainer to work on the security measure stacking and on the principle according 
	to which the attacker always seeks the easier way in. 
	\item Sometimes, the attackers consider that there is basic security in the building (locked door,
	 CCTV etc.) . In this case, it's not essential for the GM to recenter the frame. It is, however,
	  interesting to make the players think about it during the debriefing. 
	\end{itemize}
%\item	Les échanges rapides permettent un jeu vivant et ludique.
\item Fast exchange allow a living and fun game.
\end{itemize}

%\subsubsection{Les règles et objectifs des attaquants :}
\subsubsection{Attackers' rules and goals}
\begin{itemize}
%\item Un objectif simple, renvoyant aux films à gros budgets, facile à traduire
%en objectifs de sécurité informatique (entrée-sortie sans laisser de trace) ;
%\item Le budget illimité simplifie le jeu, tout en conservant la possibilité de
%discuter des aspects financiers lors du débriefing ;
%\item Le petit nombre de personnes physiques auquel a droit l'attaquant permet
%d'éviter les situations irréalistes du type « une armée de 300 personnes fait le
%siège du bâtiment » ;
%\item Le respect des lois de la physique permet encore une fois d'éviter des
%situations irréalistes et contraires à l'esprit du jeu (pas de
%téléportation/magie etc.).
\item A simple goal, sending back the players to Blockbusters, easy to translate in computer 
security goal (going in and out without leaving trace).
\item The unlimited budget simplifies the game, futhermore, it is always possible to discuss 
financial aspects during the debriefing.
\item The small number of human being authorized for the attack team during the game 
allows us to stay clear of non realistic scenario like ``laying siege with a tree hundred people army''.
\item Respecting the laws of physics allow us, once again, to stay clear of non realistic scenario or 
unsporting behavior.
\end{itemize}

%\subsubsection{Les règles et objectifs des défenseurs :}
\subsubsection{Defenders' rules and goals : }
\begin{itemize}
%\item Un objectif simple, renvoyant aux films à gros budgets, facile à traduire
%en objectifs de sécurité informatique (contrôle des entrées/sorties, ralentir
%l'attaquant etc.) ;
%\item Le budget illimité simplifie le jeu, tout en conservant la possibilité de
%discuter des aspects financiers lors du débriefing ;
%\item Le personnel illimité permet de compenser un peu le besoin de respecter
%les lois françaises, tout en permettant de faire un lien entre les mesures de
%protections parfois très chères mais inefficaces ;
%\item Le respect des lois de la physique permet encore une fois d'éviter des
%situations irréalistes et contraires à l'esprit du jeu (pas de
%téléportation/magie etc.) ;
%\item Le respect des lois du pays renvoie aux contraintes des ingénieurs en
%sécurité informatique, qui sont eux-mêmes limités par les lois du pays dans
%lequel ils pratiquent.
\item A simple goal, sending back the players to Blockbusters, easy to translate in computer 
security goal (controlling the ways in/out, slowing down the attackers etc.).
\item The unlimited budget simplifies the game, futhermore, it is always possible to discuss 
financial aspects during the debriefing.
\item The unlimited staff is here to compensate a little the need to respect the law, 
while allowing the trainees to experience that sometimes expensive security 
measures can be uneffective.
\item Respecting the laws of physics allow us, once again, to stay clear of non realistic scenario or 
unsporting behavior.
\item Respecting the laws of the country reminds the trainees that IT security 
engineers have to do the same.
\end{itemize}

%\subsubsection{Qui perd gagne}
\subsubsection{Losing and Winning}
%Il n'y a pas de perdant ou de gagnant, même si les équipes de joueurs ont
%tendance à en vouloir un. Des règles permettant de désigner des
%gagnants/perdants complexifieraient inutilement le jeu. Le but des règles est
%globalement de favoriser les échanges ludiques entre joueurs, tout en faisant
%ressortir les notions dont le formateur a besoin pour que la formation atteigne
%son but.

There is neither loser nor winner, even if the teams usually want to name one. 
Rules to define winners/losers would made the game more complex with no reason. 
The rules aim at stimulating fun exchanges between players while bringing out the idea
needed by the GM to achieve the training.

%\subsection{Animation du jeu}
%Le jeu est animé par le(s) formateur(s), nommé ``maître du jeu'' dans la suite.
%Il est important de ne pas faire de trop grosses équipes. Je conseille 2 à 3
%défenseurs et 2 à 3 attaquants. Au-delà de ce nombre, il devient très difficile
%pour le formateur de suivre les échanges et de les recadrer.
%
%Le formateur commence par expliquer l'objectif du jeu et ses règles : \\
%
%Objectif du jeu : faire prendre conscience aux élèves du travail des équipes de
%sécurité et du fait qu'ils possèdent déjà les bons réflexes : la formation doit
%leur donner les clés pour les appliquer à l'informatique.

\subsection{The facilitation of the game}
The trainer, also named Game Master (GM), facilitates the game.
It is essential to forme small teams. I recommend two to 
three defenders and the same for attackers. beyond this number, it is very 
difficult for the trainer to follow the game. 

The trainer begins the session by explaining the aim of the game, and its rules: \\

\subsubsection{Aim of the game}

Make the trainee realize that they already know security best practices.
The training is here to give them the keys to apply them to computer 
security.

%\subsubsection{Règles du jeu} 
%Il est important de bien insister sur l'aspect « physique » du jeu. Dans
%certains groupes, les élèves, conscients d'être dans une formation à la sécurité
%informatique, cherchent immédiatement à « pirater » quelque chose. Il faut
%également bien préciser que pour les deux camps, l'objectif est double
%(empêcher/réaliser le vol ou le détecter), ceci pour permettre de faire émerger
%les notions d'usurpation d'identité, de traces etc. Enfin, il ne faut pas
%hésiter à insister sur les aspects juridiques du pays: les attaquants ont tous
%les droits, mais pas les défenseurs.

\subsubsection{Game rules}
It is essential to highlight the physical aspect of the game. In a few cases
the trainees, aware that they attend a computer security training, seek straight away to 
``hack'' information systems. The double goal (prevent or detect for the defender, 
theft without being caught for the attackers) must be highlighted during the rules 
presentation, in order to make the impersonation or traces concept emerge. 
Finally, do not hesitate to insist on legal aspects: the attackers do not respect 
the rules, which is not the case of the defenders.

%\subsubsection{Déroulement du jeu}
%Dès le lancement du jeu, le formateur doit noter sur un support visible par les
%participants les différents échanges (cf. les exemples fournis dans ce
%document). En tant que Maître du Jeu, il est responsable du bon respect des
%règles, et peut limiter l'un ou l'autre des camps.

\subsubsection{Playing the game}
As soon as the game begins, the GM must write down the exchange on a medium 
visible by all players (see the example supplied in this document). As the Game Master, 
the trainer is responsible for the respect of the rules and has the right to impose limits 
to one or the other team.

%Il doit forcer les attaquants/défenseurs à détailler leurs actions dès que
%nécessaire :
He must make the players precise their action when necessary :
\begin{itemize}
%\item si quelque chose est verrouillé, on doit savoir par quel type de mesure
%(empreinte rétiniennes, empreintes digitales, badges, code, clé), et qui possède
%l'élément permettant de déverrouiller ;
%\item en cas par exemple de passage sur générateur de secours, il faut préciser
%ce que celui-ci alimente et le maître du jeu peut choisir de limiter le temps de
%fonctionnement du générateur de secours. Par exemple : un générateur de secours
%au fioul ne peut pas alimenter un système de sécurité complet d'un immeuble de
%bureau plus de quelques heures ;
%\item en cas de caméra de surveillance, il faut préciser si elles sont
%surveillées en temps réel, par qui et combien sont-ils etc.
\item if something is locked, we must now what type of lock is used (biometry -  eye or finger, 
entry pass, pin code, physical key, etc.) and who exactly owne the means to open the lock.
\item in case of generator fall back for example, the players must precise which security 
measure are supplied by the generator. The GM can limit the time during which the generator
is working. Typically if the generator supplies all the security features, it cannot work more than 
a few hours.
\item If CCTV are used, the players must specify if they are watched in real time, 
by who and by how many people. 
\end{itemize}
%La nécessité de préciser telle ou telle action est décidée par le maître du jeu,
%en fonction des enseignements qu'il souhaite tirer du jeu pendant le débriefing.
%Nous conseillons toutefois vivement de faire préciser les cas cités ci-dessus,
%ainsi que l'emplacement d'un PC footnote{Poste de Commandement} sécurité par
%exemple.
%
%Tout ce qui n'est pas explicitement dit par les défenseurs ou les attaquants
%peut être interprété/détourné par le camp adverse. Si les défenseurs n'indiquent
%jamais avoir fermé les fenêtres, les attaquants peuvent considérer qu'elles sont
%ouvertes. Si les attaquants n'ont pas dit qu'ils étaient masqués, il faut
%considérer que les caméras de surveillance filment leurs visages etc.

The need to precise one action is decided by the GM, according to the teachings 
he wants to highlight during the debriefing. However, I strongly recommend to 
make players precise the above mentionned actions.

Everything that is not explicitly said by one team can be 
interpreted/hijacked by the other team: if the defenders do not 
precise that the windows are closed, the attackers can consider them opened.
If the the attackers do not precise that they are masked, one must 
consider that their face is caught on CCTV.

%Le maître du jeu peut orienter l'un ou l'autre des camps s'il trouve que le jeu
%ne va pas dans la bonne direction,
%ou si les échanges sont laborieux. Il peut rappeler par exemple les règles au
%moment opportun, comme dire à une équipe d'attaquants timides « je vous rappelle
%que vous n'avez pas à respecter les lois françaises, vous pouvez faire exploser
%la porte/tuer le garde ». L'objectif du maître du jeu est de faire ressortir
%dans le jeu (ou de repérer) les éléments lui permettant d'illustrer, lors du
%débriefing, les principes de base de la sécurité informatique.
%
%Aucune comparaison/lien avec la sécurité informatique ne doit être réalisé
%pendant le jeu de rôle. Les liens sont effectués lors du débriefing.

The game master can guide one or the other team if he thinks the game
is not going in the right direction, or to revive it. He can, for example, 
bring back the rules at the appropriate time, like saying to a 
shy attack team ``I remind you that you do not need to follow the law, you
can blow up this doors or kill this guard''. The GM's goals is to bring up in the game 
(or look for) the ideas allowing him to illustrate the basic principles of computer 
security during the debriefing.

No analogy with computer security must be done during the game. 
The link is brought up during the debriefing only.

%\subsubsection{Fin du jeu}
%Il est conseillé de terminer le scenario en cours dans les cas suivants : 
\subsubsection{Game over}
It is recommended to close the ongoing scenario if: 
\begin{itemize}
%\item Les attaquants s'entêtent dans une direction alors que d'autres
%possibilités n'ont pas été explorées ;
%\item	Le scenario en cours devient trop complexe ; 
%\item	Le scenario en cours devient irréaliste ; 
%\item	Le formateur souhaite inverser les équipes ; 
%\item Les joueurs perdent en motivation (il est alors possible d'arrêter le jeu
%ou d'inverser les équipes)
%\item	Le formateur a déjà toute la matière dont il a besoin pour son débriefing.
\item The attackers keep going in the same unsuccessful course of action; 
\item The ongoing scenario becomes too complex;
\item The ongoing scenario becomes too unrealistic;
\item The trainer wish to switch teams;
\item The players start to lose motivation
(it is then possible to either stop the game or switch teams);
\item The trainer already has the material he needs for the debriefing.
\end{itemize}

%\subsection{Exemple d'échange/scenario}
\subsection{Exchange/scenario example}

%Cet exemple a été observé au cours d'une des formations, à ce moment, le jeu
%durait déjà depuis 10 minutes.

This exchange was observed during a training. At this time, the game was on for 
10 minutes.

\begin{longtable}{|p{4cm}|p{4cm}|p{4cm}|}
%\caption[Exemple de scenario]{Exemple de scenario}\\
\caption[Scenario example]{Scenario example}\\

\hline
%Attaquants & Défenseurs & Maître du jeu \\
Attackers & Defenders & Game Master \\
\hline
\endhead

\hline \endfoot
          
%Corruption de sous\-traitants pour qu'ils réalisent eux-mêmes le vol & & \\ 
Corrupt a subcontractor's employee and make him carry out the theft && \\
%&L'objet en utilisation reste en visibilité permanente de son utilisateur. Dès
%qu'il n'est plus utilisé, il est rangé dans un coffre-fort fermant à clé. Trois
%personnes ont chacune une clé: Le responsable du service de l'utilisateur de
%l'objet, l'utilisateur de l'objet et le directeur de la sécurité. On trace les
%actions des responsables de ces trois clés en permanence. & Qui possède la clé
%du coffre ? \\
& When used, the object stay visible to the user at all time. As soon as the user
 has finished, the object is put in a safe locked up by a physical key. Three person have 
a copy of the key : the user himself, his manager and the company's head 
of security. The actions of the keys owner are tracked. &
Who have the key of the safe ? \\
%Récupération du nom du directeur de la sécurité, phase d'observation pour
%connaître son emploi du temps. Vol avec violence permettant de récupérer la clé
%et de la donner au sous-traitant & & \\
Find the name of the company's head of security, watch his schedule. 
Violent theft of the key witch is then given to the subcontractor. && \\
%& Le coffre n'est pas en évidence & Mesure non effective : Le personnel
%d'entretien peut le trouver facilement.\\
& The safe is not easily found & 
Uneffective measure: the maintenance staff can find it easily \\
%& Vidéo Surveillance multi-écran. Un gardien 24/24 pour surveiller les écrans et
%un enregistrement. Un gardien est également présent à l'accueil & Attention,
%trop de caméras implique difficulté/impossibilité pour un humain de les
%surveiller en temps réel \\
& CCTV on multiple surveillance screens. One guard is behind the screens 24/7,
the video streams are recorded. Another guard is in the lobby. & 
Warning : to many cameras implies it is difficult to watch them in real time \\
%Une femme de ménage distrait le gardien de la vidéo surveillance pendant que
%l'autre commet le vol & & \\
A cleaning lady distracts the CCTV guard while another one perpetrates the theft && \\
%& Formation accrue des gardiens (par la police, les forces spéciales etc.),
%enquête de moralité des sous-traitants & Il sera toujours possible de trouver un
%sous-traitant « faible », le gardien pourrait être malade, avoir besoin d'aller
%aux toilettes etc. Mais les attaquants ont « perdu » ! Les enregistrements vidéo
%sont revus et contiennent le visage de la femme de ménage \\
& The guards were trained by the special forces, there is a background check on all
subcontractors. & There is always a way to find a weakness to exploit to blackmail a personn. 
Futhermore, the guards could need to go to the bathroom, or can be sick. But the attackers loose: 
the cleaning lady's face is caught on CCTV. \\
%La femme de ménage se déguise dans les toilettes/La personne distrayant le
%gardien utilise un appareil permettant de détruire à distance les données sur
%disque dur (aimant) & & \\
The cleaning lady hides in the bathroom to dress up, the person distraying the CCTV guards 
uses a device that can destroy the video data on hard drive (magnet) && \\
%& Vidéo-Surveillance dans le couloir devant les toilettes et salle serveur
%protégée (au centre de l'immeuble, avec cage de faraday). & La caméra a été mise
%devant les toilettes suite à un recadrage du Maître du Jeu, la loi interdit de
%mettre des caméras dans les toilettes ! \\
& There is a CCTV camera on the corridor leading to the bathroom, the server room is
protected against tampering (in the center of the building, in a faraday cage) &
The CCTV camera has been put in front of the bathroom instead of inside it because 
of a GM remark, french law does not allow CCTV in bathrooms. \\
%Débrancher la caméra & & \\ 
Unplug the CCTV in front of the bathroom & & \\
%& Alarme sonore et visuelle se déclenchant dans la loge du gardien en cas de
%dysfonctionnement/débranchement de caméra & Le Maître du Jeu déclare la fin du
%scenario, pour forcer l'équipe d'attaquant à passer à autre chose.\\
& Audio and visio warning in the guard lodge as soon as the camera is unplugged 
or malfunctionning & The Game Master forces the end of the scenario, to make attackers 
move on. \\
\end{longtable}

%\section{Le débriefing du jeu de rôle}
%\subsection{Apprentissage des bons réflexes « communs » } 
%Comme indiqué en introduction de ce document, les personnes néophytes 
%ont déjà de bons réflexes, qui peuvent être appliqués aussi bien en sécurité
%physique qu'en sécurité informatique. Je conseille de présenter ces reflexes en
%sortie de jeu de rôle, en faisant le lien avec les scenarii apparus durant le
%jeu. Voici une liste non-exhaustive de bons réflexes à mettre en avant par le
%formateur :

\section{The game's debrief}
\subsection{Learning the common basic good practices}
As explained in the introduction, neophyte people already know  security 
good practices that can be apply to physical security as well as to computer 
security. I recommand to present these good practices just after the game, in 
order to link them to the scenarii come up during the game. You can find below 
a non-exhaustive list of good practices needed to be hightlighted by the trainer : 

\begin{itemize}
%\item	Ne pas « faire confiance » par défaut ;
%\item On vérifie les identités ;
%\item On ne donne pas la clé de sa maison/son code d'alarme/mot de passe à
%n'importe qui
%\item	Cas des services d'urgence : donnez-vous votre clé « au cas où »?
%\item	Pourquoi donnerait-on son mot de passe au SAV ?
%\item On appelle la police/les services de sécurité en cas de suspicion
%d'activité malveillante
%\item	On se pose les questions : 
%\item	« Quelqu'un a-t-il intérêt à attaquer mon bâtiment ? », « À quel point ? »
%\item	Cette information/clé/badge serait-il utile à quelqu'un ?
%\item	Que se passe-t-il en cas de dysfonctionnement ?
\item Do not trust by default;
\item Check IDs;
\item Don't give your home key/alarm pin/password to anybody;
\item Case of the emergency services : would you give them your home key ``in the event of'' ?
\item Call the police/security team when you suspect malicious activity ;
\item Aask ourself : 
\item Could someone be interested in attacking my building ? To which extent ?
\item Could this information/badge/key be of value to someone ?
\item What do I do in case of malfunction ?
\end{itemize}

%\subsection{Grille de lecture des scenarii}
%
%Les éléments utilisés lors du jeu de rôle par les différents participants
%ont une correspondance facile avec la sécurité informatique. L'idée du
%débriefing est de réaliser ce lien entre le jeu et les aspects de la sécurité
%informatique sur lesquelles le formateur souhaite insister. La figure 1 présente 
%une grille de lecture (non exhaustive) des éléments généralement utilisés par les joueurs et
%de ce qu'ils peuvent représenter dans le milieu de la sécurité informatique.

%%%j'en suis là

\subsection{Scenarii decoding keys}
It's easy to make a connection between the physical element used by the trainee during 
the game and computer security elements. The debrief idea is to have the trainer making
this connection, according to the key points he wants to highlight. The table 1 presents a 
non-exaustive list of decoding keys of wildly appearing elements in the game: 

\begin{figure}
    \begin{tabularx}{\textwidth}{|X|X|}
    \hline
%Sécurité Physique & Sécurité Informatique \\ \hline
Physical security & Computer security \\ \hline
%Clé/Badge & Mot de passe/carte à puce \\
Key / Badge & Password, smartcard \\
%Coffre/Porte blindée & Mesure technique de sécurité \\
Safe, reinforced door & technical measure of protection \\
%Vidéo Surveillance & Supervision/logs/anti-virus \\
CCTV & Supervision/logs/anti-virus \\
%Destruction des enregistrements de vidéo surveillance & Destruction/Altération
%des logs \\
CCTV redcords destruction & Logs destruction or tampering \\
%Coupure d'électricité/incendie	                        & Denis de Service \\    
Blackout / arson & Denial of Services \\
%Gardes/gardiens/personnel	                            & Opérationnels \\
Guards, surveillance employee & Security Operationnals \\
%Déguisement/fausse carte d'identité & Usurpation d'adresse IP, usurpation
%d'identité \\
Disguise/false ID card & Impersonnation of IP adresses or identity \\
%Observation, récupérer le nom d'un chef, une information … & Social Engineering
%\\
Observation, get some top manager's name, get info ... & Social Engineering \\
%Procédure d'urgence, générateur de secours, etc. & Résistance aux pannes,
%défense en profondeur, procédure SAV \\
Emergency procedure, generator etc. & Failure resistance, in-depth security, after sale \\
%    Carte d'identité 	                                    & Certificats \\
ID card & Certificate \\
%Utilisation par les attaquants d'une technologie spécifique (brouilleur,
%explosifs, drones etc.) & Utilisation d'exploit/de plateforme écrits par
%d'autres \\ \hline
Specific technology use (jammer, explosive, drone ..) &
Use of exploits, command and control center etc. \\ \hline
    \end{tabularx}
   % \caption{Grille de Lecture}
    \caption{Decoding keys}
\end{figure}

%\subsection{Points Communs et Divergences}
%
%Les points communs entre les sécurités physique et informatique ont déjà été
%présentés plusieurs fois au cours de ce document, nous les reprenons maintenant
%et indiquons les exemples types illustrant ces principes et apparaissant dans le
%jeu de rôle.
\subsection{Similarities and divergences}
The similarities between physical and computer security have already 
been presented multiple times in this document. We now get over them one 
more time to highlight key examples that illustrate these principles and come up 
in the role playing game.

%\subsubsection{La problématique de « faire confiance » à une entité/personne }
%Très vite dans le jeu, les participants sont confrontés à la notion de contrôle
%d'accès. Vous verrez assez vite apparaître des notions de badge/vérification de
%cartes d'identité à l'accueil, ou d'attaquant se déguisant ou mentant pour
%accéder à l'immeuble. Il est important d'utiliser ces points pour faire
%réfléchir les participants à la notion de confiance, d'identité et
%d'authentification. L'utilisation d'une fausse carte d'identité par l'attaquant
%est par exemple intéressante : qu'est ce qui nous permet de croire quelqu'un
%quand il décline son identité ? Cette notion est centrale dans tout système de
%sécurité. Le formateur peut également profiter de cette discussion pour parler
%des différentes possibilités :
\subsubsection{The ``trusting someone'' problem}
Very quick in the game, gamers are exposed to the access control principle. 
You'll see appear quickly the concept of badges, ID verification in the lobby or 
disguised or lying attackers. It's important to use this key points to make the 
trainee think about the concepts of trust, identity and authentication. The use 
of a false ID card is, for instance, very interresting : what can we use to trust 
someone when he states his identity ? This notion is at the center of every 
security system. The trainer can also take advantage of this discussion to talk 
about the different authentication methods : 
\begin{itemize}
%\item	Biométrie
%\item	PIN ou mot de passe
%\item	Clé (qui peut être volée, perdue, copiée etc.)
%\item Carte d'identité, qui renvoie à la notion de faire confiance à une tierce
%partie (l'état dans la sécurité physique, une autorité de confiance dans la
%sécurité informatique)
\item Biometry
\item PIN code or passwords
\item Key (whitch can be lost, stolen, copied etc.)
\item ID cars, whitch sends back to the concept of trusting a third party 
(the government in physical security, the Certification Autority in Infosec)
\end{itemize}

%Enfin, dans la plupart des sessions effectuées, les attaquants utilisent assez
%vite des mensonges/usurpation d'identité pour contourner des mesures de
%sécurité. Par exemple, dans une session, les attaquants récupéraient le nom d'un
%manager haut placé, et insistaient sur une livraison urgente à ce dirigeant à
%l'accueil. Ce genre de scenario est très utile pour illustrer le concept de
%social engineering. C'est enfin l'occasion de faire réfléchir les élèves sur la
%maxime « l'élément le plus faible est l'humain ».
Finally, in most game sessions, the attackers were fast using lies or identity impersonation.
For example, in one of the session, the attackers were geting the name of a top managers, 
and were insisting on the urgent nature of a delivery at the reception. This type of scenario 
is very usefull to illustrate the concepts of phishing, scam and social engineering. 
It's also the moment to make the trainee think about a great principles in security 
``the human is the weakest part''. 



%\subsubsection{La notion de « défense en profondeur »}
%Le principe de défense en profondeur, qui consiste à empiler les couches de
%sécurité et à traiter les cas où une couche est défaillante, apparait facilement
%lors du jeu de rôle. Par exemple, de façon systématique, les élèves proposent un
%contrôle d'accès à l'entrée de l'immeuble, puis un contrôle d'accès accru à la
%pièce dans laquelle est stocké l'objet. Ils peuvent même rajouter un contrôle
%supplémentaire autour de l'objet dans cette pièce.

\subsubsection{In depth security}
The in depth secuoity idea, which consist in piling up security measures and 
handling the possible failure of one of them, appears easily in the game. 
For example the trainee consistensly proposed an access control in the lobby
and a different one for the room where the object is stored. Often, they 
even added an access control near the object itself.

%Le formateur se doit de mettre en avant ce comportement, et de faire remarquer
%aux élèves qu'il en est de même dans la sécurité informatique. C'est le moment
%de discuter avec eux des mesures de sécurité multiples, et de faire prendre
%conscience de leur intérêt. Nous entendons souvent, en tant qu'ingénieur en
%sécurité informatique des phrases du type « mais c'est dans le LAN, nous ne
%risquons rien », « mais là nous avons déjà tapé un mot de passe une heure avant,
%pourquoi un autre ? » Etc.
The trainer must highlight this behaviour, and make the trainee notice that the 
same applies to computer security. It's the moment to talk with them about multiple 
security measures, and to make them aware of their convenience. We often hear, 
as security engineers, sentences like ``But it's in the LAN, ther is no risk'' or 
``but the user has already enter another password, why do we need a new one ?'' etc.

%La multiplication du type de technologie (clé physique, badge, biométrie etc.)
%est aussi une façon de faire réfléchir les élèves sur les règles d'hygiène
%informatique (une clé/mot de passe par usage, etc.).
%Enfin, les différentes tentatives des attaquants permettent d'illustrer très
%bien la règle d'or selon laquelle le niveau de sécurité d'un système dépend du
%niveau de sécurité de son composant le plus faible.
The multiplication of technologies (physical key, badge, biometry etx-c.) is also a way 
to make trainee think about the security best practice (one password per usage etc.).
Finally, the attackers' different attemps allow us to illustrate the fact that the security level
of a system depends on the security level of its weakest element.

%\subsubsection{Les motivations de l'attaquant}
%Les différents scenarii permettent au formateur d'illustrer la notion importante
%de motivation de l'attaquant (et du défenseur). Lorsque nous arrivons à des
%scenarii qui représentent plusieurs millions de d'euros, et des mois de
%préparation, la question se pose : l'objet en vaut-il la peine ? La même
%question peut être posée pour les défenseurs.
\subsubsection{The attackers' motivation}
The differents scenarii allow the trainer to illustrate the important notion of the 
attackers' (or defenders') motivations. When the attack itself cost millions and months of 
preparation, we can ask ourselves : is the object worth it ?
The same question may be asked to the defenders.

%C'est également le moment de discuter du niveau de sécurité par rapport au
%niveau de l'attaquant, et de faire réfléchir les élèves aux questions au cœur de
%tout système de sécurité : que protège-t-on et contre qui ?
It is also an opportunity to discuss the security level versus the attackers level, and to 
think about the question at the earth of all security systems : 
what do we protect, and against who ?

%\subsubsection{La démystification de l'attaquant (qui n'est pas un « hacker de
%génie »)}
%Une des notions les plus mal perçues par les néophytes est la diversité des
%profils d'attaquants informatique. L'imaginaire collectif dépeint une image de «
%génie » au fond d'une cave, or, comme en sécurité physique, il y a plusieurs
%types d'attaquants : si la porte ne ferme pas à clé, n'importe quel délinquant
%peut entrer dans l'immeuble. Lorsque le scenario devient complexe, nous sommes
%face à des attaquants extrêmement motivés ciblant un objectif bien défini.
\subsection{Demistify the attackers (who is not a computer genius)}
One of the idea the least understood by a neophyte audience is the diversity 
of the attackers profiles. The collective imagination depicts them as a genius hackers, 
in an undergournd cave, yet, as in physical security, there is a variety of attackers:
if your door is not locked, every delinquent can enter your building. When the scenario
becomes complex, we face very well organized and motivated attackers.

%La notion d'économie souterraine est également mal comprise : 
The blackmarket idea is also not well understood
\begin{itemize}
%\item En sécurité physique les objets sont revendus ou « commandés » avant le
%vol. La même chose existe en sécurité informatique et les élèves doivent en
%prendre conscience ;
%\item De même qu'un attaquant « physique » va acheter des outils lui permettant
%de réussir son attaque (explosifs, brouilleurs radio, fausses carte d'identités
%etc.), l'attaquant informatique fait de même. Ce qui veut dire qu'il y a une
%économie liée à la découverte de ces outils (failles, exploit etc.) et à leur
%revente. C'est le moment de faire réfléchir les élèves sur les différents
%profils. N'importe qui peut appuyer sur le bouton d'un brouilleur radio, il faut
%en revanche des compétences techniques poussées pour le concevoir.
\item In physical security, the objects are resold or ordered prior the theft. It is the
same in computer security, and the trainee must be aware of this. 
\item As a physical attackers will buy specific tools (explosives, jammers, 
false ID, ...), and computer attacker will do the same. Which means an economy  
has developped ourond the discovery of tools (vulnerabilities, exploit etc.) and their trade. 
Make the trainee aware of these different profiles : anybody can 
push a button on a jammer, but you need specific skills to design one.
\end{itemize}


%\subsubsection{Les notions de compromis entre contraintes opérationnelles et
%sécurité }
%Pour cette notion, le formateur doit se concentrer sur les mesures mises en
%œuvre par les défenseurs, et des contraintes qu'elles impliquent pour les
%employés de la société ou la société elle-même. Le lien est alors assez facile à
%faire avec les contraintes de sécurité informatique.
\subsubsection{The constraint versus security compromise}
To illustrate this idea, the trainer must focus on the security measures 
deployed by the defenders, and the contraints they imply for the company's 
employees or the company itself. The link is then easily made with computer 
security contraints.

%Une partie intéressante à travailler est la présence dans les scenarii d'équipe
%de secours (police, armée, pompiers…) légitimes ou non. Demander aux élèves :
%donnent-ils accès à tout l'immeuble de façon systématique à toutes les équipes
%de secours « au cas où » ? Vérifient-ils la légitimité des services de secours ?
%Dans une des sessions, les attaquants s'étaient faits passer pour une équipe
%médicale d'évacuation par hélicoptère pour extraire l'objet volé. C'est le
%moment de discuter des accès SAV/service informatique etc. Des stockages de mot
%de passe en clair « au cas où le client l'oublie etc. ».
One interesting element to work on is the presence of emergency services (police, 
army, firefighters, etc.) whether they are legitimate or not. Ask the trainees: 
do they give the emergency teams full access to the building, just in case? Do they check if they 
are legitimate? In one of the game sessions, the attackers
posed as a medical team who evacuate victims via helicoper (they, in fact, were 
evacuating the stolen object). This is the time to discuss the privilege accesses 
of teams like after sales, IT support etc. and the need to store cleartext passwords 
``in the case of the client needs it''.

%\subsubsection{Les objectifs des équipes de sécurité (prévoir le comportement de
%l'attaquant, le prévenir ou le détecter)}
%Dans le jeu de rôle, le travail des défenseurs est facilité par rapport aux
%conditions réelles : les attaquants annoncent leur intention et leur objectif
%est connu. Le formateur peut profiter du débriefing pour mettre le doigt sur la
%difficulté des équipes de sécurité : elles doivent imaginer le comportement des
%attaquants et évaluer leurs possibles motivations. 
%Le formateur peut également faire réfléchir sur les outils de supervisions de
%traces. Lorsqu'il n'est pas possible de prévenir le comportement d'un attaquant,
%il faut au minimum le détecter.
\subsubsection{The security teams' goals (predict the attacker behavior, 
prevent or detect it)}
In the game, the work for the defenders team is easier than in the real world : 
the attackers announce their intention and their goal is known. The trainer can pinpoint, 
during the debrief, the difficulties of the security teams' work, they have to imagine the 
attackers' behavior and evaluate their possible motivations.
The trainer can also make the trainees think about supervision or tracing tools. 

%\subsubsection{Divergences}
\subsubsection{Divergences}

%Toute la sécurité physique n'est, bien sûr, pas transposable en sécurité
%informatique (et vice versa). Mais les différences, aussi essentielles
%soient-elles, sont finalement peu nombreuses sur le fond :
Of course, the whole physical security isn't transposable into infosec (and vice versa). But 
the differences, as essential as they may be, are not that many:
\begin{itemize}
%\item	Le facteur temps diffère énormément : 
\item The time factor differs greatly:
\begin{itemize}
%\item Par exemple : tester un mot de passe est bien plus rapide que tester une
%clé sur une serrure physique
\item In example : testing a password is a lot faster than testing a physical
key on a door. 
\end{itemize}
%\item Le facteur géographique n'existe pratiquement plus :
\item The geographic factor nearly no longer exists:
	\begin{itemize}
%\item L'attaquant n'a aucun besoin d'être physiquement présent pour mener
%l'attaque. La distance géographique n'a pas d'importance.
\item The attacker does not need to be physically present to conduct the attack. 
The physical distance does not matter anymore.
	%\item	Il y a des exceptions à cette règle : 
	\item There is, of course, exceptions to this rule:
		\begin{itemize}
%\item Les lois qui s'appliquent sont liées à la localisation physique des
%données volées ou altérées ;
\item The laws depend on the physical location of the stolen or 
tampered data;
%\item Lors d'attaques de type signaux compromettants, radio ou hardware, la
%distance géographique peut redevenir un élément déterminant ;
\item When attacking via compromising signals, radio flux or hardware element, 
the physical distance can come up again as a critical issue.
		\end{itemize}
	\end{itemize}
%\item Ces deux changements d'échelle rendent les attaques de masse peu coûteuses
%et accessibles à tous ;
\item These two scale changes result in mass attacks costing less and put them
within anybody's reach;
%\item Les traces exactes et facilement récupérables concernent les machines et
%plus difficilement les personnes physiques
\item The exact and easily collected evidences only relate to the machines, less easily to the human beings;
	\begin{itemize}
%	\item	Il peut être très difficile de remonter au coupable ;
	\item It can be very difficult to find the actual perpetrator;
%\item L'attaquant peut se cacher derrière des machines de tierces parties non
%concernées ;
	\item The attackers can hide themself behind innocent third parties;
	\end{itemize}
%\item	Le vol est pratiquement impossible à détecter (copie informatique)
\item The theft is virtually impossible to detect (electronic copy);
	\begin{itemize}
%\item Les traces du vol peuvent être retrouvées si le système est correctement
%configuré ;
\item Some evidences of the theft can be found is the system is correctly configured;
	\end{itemize}
%\item Encore trop souvent, aucune sécurité basique n'est mise en place en
%informatique, alors que dans le monde physique, les gens mettent en œuvre au
%minimum une porte qui ferme à clé.
\item Too often, there is no basic security deployed in IT, where, in the physical world, 
people would have a working lock on the door, at a minimum.
\end{itemize}

\newpage
%%%%%%%%%%%
%\section {Annexe : exemple d'une session}
%Cette session a été réalisée avec cinq personnes (trois défenseurs, deux
%attaquants), elle a duré environ 50 minutes.
\section{Game session example}
This game session has been carried out with five people (three defenders and two attackers), 
it lasted nearly fifty minutes.

\begin{longtable}{|p{3cm}|p{3cm}|p{3cm}|p{3cm}|}
    
%\caption[Cas réel]{Exemple d'une session complète à 5 joueurs d'une durée de 50
%minutes environ (hors débriefing)}\\

  \caption[Real case]{Example of a full game session with five player for a duration of 
  nearly fifty minutes (without debrief)}  \\
      
\hline 
%Attaquants	& Défenseurs	& Commentaires	 & Parallèle sécurité informatique \\
Attackers & Defenders & Comments & IT security parallel \\
\hline 
\endhead 
\hline 
\endfoot

%Ouvre la porte, va chercher l'objet, ressort avec l'objet & & & Vol de données
%non protégées \\
Open the door, collect the object, get out & & & Unprotected data theft \\
%& La porte est sécurisée par badgeuse et est fermée complètement après 20h. Si
%tentative d'effraction, alarme relié au commissariat & & Protection par mot de
%passe, gestion des droits utilisateurs, supervision \\
& The door is secured by a badger and is physically locked after 8 PM. 
If an attempted theft is detected an alarm is trigerred, linked directly to the 
police station & & Password based protection, access control, supervision \\
%Une femme séduit un employé et lui explique qu'elle a oublié son badge,
%l'employé la fait entrer (vol puis sortie) & & & Social engineering \\
A woman is sent to seduce an employee, she tells him she has forgotten her badge,
the man employee let her pass (theft then exit) & & & Social engineering \\
%& L'objet est enfermé dans un coffre-fort à code. Seul le responsable du site
%dispose du code (il faut l'appeler pour utiliser l'objet). Le port du badge
%apparent est obligatoire dans l'entreprise, des agents de sécurité vérifient
%l'application de la consigne. Les salariés sont de plus sensibilisés sur le
%danger de laisser entrer un inconnu. 
& The object is locked in a safe, a PIN code is needed to open the safe. The 
site supervisor is the only one who knows the PIN (people must call him each time
they need to use the object). Carrying a visible badge is mandatory within the building,
security agent ensure the enforcement of the rule. Futhermore, employees
are aware of the risks lying in letting an unknow personn enter the building.
%& On remarque que la mesure est très
%contraignante pour l'entreprise (une seule personne pour l'accès à l'objet qui
%doit être utilisable) & Protection par mot de passe. Non partage des mots de
%passes. Logs et supervision. Sensibilisation \\
& We can notice that the measure is very restrictive for the company (one and 
only one personn has access to the object) & Password based protection. Non
sharing of passwords. Supervision. Awareness training.\\
%Déguisement en personnel d'entretien, entre avec un badge volé et un chariot qui
%contient un chalumeau. Ouvre le coffre au chalumeau, récupère l'objet, le met
%dans le chariot et ressort. & & & Usurpation d'identité, attaque par force brute
%\\
Dressing up as a janitor, entering with a stolen badge and a cart containing a 
blowtorch. Open the safe with the blowtorch, get the object, put it in the cart
and exit & & & Impersonation, brute force attack \\
%& Il y a un détecteur de fumée dans la pièce. L'accès à la pièce est protégé par
%un système de reconnaissance rétinien. & & Détection d'attaque. Biométrie \\
& There is a smoke sensor in the room. The entry of the room is protected by 
a retinal scan. & & Attack detection, biometry.\\
%L'attaquant se pose en hélicoptère sur le toit de l'immeuble, puis utilise les
%conduits de climatisation jusqu'à la pièce. Descente "à la mission impossible"
%et vol du coffre. Sortie de l'immeuble, ouverture du coffre etc. & & & Attaque
%"Hors ligne", vol puis tentative de contournement de la mesure de sécurité. \\
The attackers land a helicopter on the roof of the building and use the air conditionning 
pipes to gain access to the room. Go down "like in 'Mission : Impossible'" and 
steal the safe. Exit from the building and then open the safe. & & & OffLine
attack, theft followed by protection workaround. \\
%& Le coffre est scellé dans le mur, il est de plus électrifié si le système de
%reconnaissance rétinien n'a pas validé l'entrée dans la pièce. & & Interdiction
%d'attaque "hors ligne". Interdiction de toute action sans mot de passe \\
& The safe is sealed in the wall, futhermore, it is electrified until the retinal scan
is OK & &  Offline attacks banning. Ban all action before authentication check. \\
%Coupure d'électricité & & & Denis de service/panne du système de sécurité \\
Blackout & & & Denial of Service/failure of the security system \\
%& Groupe électrogène de secours maintenant les fonctions de sécurité de la pièce
%& & Système de secours \\
& Generator supplies all the security measure of the room & & emergency back-up 
system \\
%L'attaquant fait chanter un employé ayant les accès à la pièce en prenant en
%otage sa famille. L'employé réalise le vol & & & Social engineering \\
The attackers take a member of an employee's family in hostage and blackmail him to
commit the theft himself. & & & Social engineering \\
%&Des caméras sont présentes devant et dans la pièce, elles sont surveillées
%24/24h par du personnel au PC Sécurité qui se trouve à l'extérieur du bâtiment.
%& & Logs et supervision sur des serveurs dédiés \\
& CCTV camera are placed in front of and in the room, the camera feeds are 
watched in real time 24/7 by employees in the security command center which is not 
in the same building. & & Supervision and logs on dedicated servers \\
%L'attaquant fait exploser le PC sécurité & & & Destruction, modification des
%logs \\
Blowing up the security command center & & & Detruction, tampering of the logs \\
%& En cas d'explosion ou de coupure du PC sécurité, des équipes sont envoyées au
%PC sécurité et dans l'immeuble. Une alarme se déclenche en cas de coupure du
%lien de communication entre PC sécurité et batiment principal & & Protection des
%logs, défense en profondeur (mécanisme de secours du mécanisme de secours) \\
& In case of explosion or communication loss with the security command center, 
teams of guards are sent to the command center and to the building. An alarm is trigered 
in case of communication loss. & & Logs protection, in depths security, monitoring of security
measures etc. \\
%Piratage des caméras de sécurité pour couper le flux vidéo & & & Attaque du
%système écrasant les logs \\
Hacking of the CCTV feed to cut the video stream & & & Attack to destroyed the logs, 
DoS on the supervision system \\
%&Un capteur de mouvement est placé sur l'objet, en cas de mouvement, l'objet
%explose. & Refusé : l'objet doit être utilisable pendant la journée. non
%conforme avec la loi française. & En informatique, on appelle cette technique
%l'effacement d'urgence. En cas d'attaque détectée, les données sensibles sont
%effacées. Très contraignant \\
& A motion sensor is put on the object, if triggered, the object blows up. & Refused : the
object must be usable during the day, and not compliant with french law & In InfoSec, 
we call this technique ``emergency erase''. If an attacked is detected, all sensitive data
are erased. Very constraining. \\
%Intrusion en passant par les champs morts de la caméra et vol d'un badge pour
%entrée & & & \\
Intrusion by using the CCTV camera blind spots, theft of a badge for entering & & & \\
%& Tous les angles de caméras sont couverts et la surveillance se fait par une
%personne par caméra& Mesure très chère en personnel & Augmentation de la
%supervision et du personnel opérationnel \\
& Enough CCTV camera to have no blind spot at all, there is one watching
 guards per screen, one screen per camera. & Costfull measure 
 & Increase of the supervision and security operationnals \\
%    Cache caméra montrant une photo de l'emplacement	 	& & & \\ 	 
Cover the camera with a picture of the hallway & & & \\
%& Un garde est présent à l'accueil et contrôle les entrées, des rondes sont de
%plus effectuées avec chiens de garde & & \\
& A guard is in the lobby and control all the entry, patrols with dogs && \\
%L'attaquant attaque les gardes et nourri les chiens pour les distraire & & & \\
Kill the guards and feed the dogs to distract them & & & \\
%&Toujours un contrôle rétinien, un ajout de contrôle d'empreinte pour accès à la
%salle. & & Contrôle d'accès biométrique \\
& There is always the retinal scan, a fingerprint scan is added. & & Biometry \\
%    Salarié infiltré (taupe) qui réalise le vol	 	& & & \\ 	
 An infiltrated employee commit the theft & & & \\
%&Fouille systématique des employés du site & Mesure très contraignante
%(plusieurs secondes par employés, en horaire de pointe …) & Contrôle permanent
%de l'ensemble du contenu des ordinateurs des employés (interdit) \\
& Systematic personn search at each going in/out the building &
Very contraining measure (several seconds by person, in rush hour, etc.) &
Real time control of everything stored on employees computers (forbidden by 
french law) \\
%Utilisation d'un drone pour faire sortir l'objet & & & Exfiltration de données
%\\
Drone use to get the object out of the building & & & Data exfiltration \\
%&Fouille à l'entrée/sortie de la pièce. & Mesure très contraignante (plusieurs
%secondes par employé, en horaire de pointe …) & \\
& Personn search at each going in/out the room. & Very constraining measure
 (several seconds by personn, in rush hour ...) & \\
 %   Meurtre du garde surveillant la pièce & & & \\	 
Murder of the guard securing the room & & & \\	 	 
%&Un antivol permet de détecter que l'objet est sortie du bâtiment, le site est
%verrouillé & & Watermarking des données (moins efficace) \\
& Antitheft device on the object allows to know when the object leaves the building, 
in case of detection, the site is lock down. & & Data watermarking (less effective) \\
%Déclenchement d'un incendie pour ouverture automatique des portes d'évacuation &
%& & Attaque des procédures de secours/d'urgence \\
Trigger an arson to obtain the automatic opening of the doors & & &
Emergency procedure attack \\
%        & Géolocalisation de l'objet 	 & & \\	
& Geolocalization of the object & & \\
 %   Utilisation de Papier aluminium pour empêcher la détection 	  & & & \\	 	
 Use of silver foil to avoid detection & & & \\
 %       & Intervention de l'armée pour abattre le drone 	 & & \\	 
 & Army intervene to take down the drone & & \\
 %   Plusieurs centaines de drones font diversion	 	& & & \\ 	 
 Hundreds of drone making diversion & & & \\
    %    &Brouilleur radio pour empêcher le pilotage des drones	 	& & \\ 
& Radio jammer to prevent piloting the drones & & \\
 %   Drones autoguidés, préprogrammés	 & & & \\	 
 Drones autopilot pre programmed & & & \\	 
  %      & Brouillage des signaux GPS	&  & \\ 	
  & jammer for GPS signals to prenvent the autopilot working & & \\
%Passage par le sous-sol pendant que le drone sort un leurre de l'objet, sortie à
%3 motos, seule l'une à l'objet & & & Diversion/surcharge de la supervision \\
Passing by the underground parking while the drone gets out with a copy 
of the object, exit on three motocycles, only one has the object & & &
Diversion, overload of the supervision system \\
%&Des clous ont été mis sur a route lors du déclenchement de l'alarme. Porte
%blindée en sortie de parking & & Toutes les notions attendues ont été exprimées,
%de plus, le scenario devient trop complexe. Arrêt du jeu. \\
& Nails were spread on the exit road as soon as the alarm was triggered.
There is a reinforced door at the exit of the parking lot & & All expected ideas have been 
expressed, futhermore, the scenario is becoming to complex. End of the game.
\end{longtable}

\end{document}
