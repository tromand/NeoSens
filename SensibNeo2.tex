% !TEX TS-program = pdflatex
% !TEX encoding = UTF-8 Unicode

% This is a simple template for a LaTeX document using the "article" class.
% See "book", "report", "letter" for other types of document.

\documentclass[11pt]{article} % use larger type; default would be 10pt

\usepackage[utf8]{inputenc} % set input encoding (not needed with XeLaTeX)
\usepackage[french]{babel}

%%% Examples of Article customizations
% These packages are optional, depending whether you want the features they
%provide.
% See the LaTeX Companion or other references for full information.

%%% PAGE DIMENSIONS
\usepackage{geometry} % to change the page dimensions
\geometry{a4paper} % or letterpaper (US) or a5paper or....
% \geometry{margin=2in} % for example, change the margins to 2 inches all round
% \geometry{landscape} % set up the page for landscape
%   read geometry.pdf for detailed page layout information

\usepackage{graphicx} % support the \includegraphics command and options

% \usepackage[parfill]{parskip} % Activate to begin paragraphs with an empty
%line rather than an indent

%%% PACKAGES
\usepackage{booktabs} % for much better looking tables
\usepackage{array} % for better arrays (eg matrices) in maths
\usepackage{paralist} % very flexible & customisable lists (eg.
%enumerate/itemize, etc.)
\usepackage{verbatim} % adds environment for commenting out blocks of text & for
%better verbatim
\usepackage{subfig} % make it possible to include more than one captioned
%figure/table in a single float
\usepackage{longtable}
\usepackage{tabularx}
% These packages are all incorporated in the memoir class to one degree or
%another...

%%% HEADERS & FOOTERS
\usepackage{fancyhdr} % This should be set AFTER setting up the page geometry
\pagestyle{fancy} % options: empty , plain , fancy
\renewcommand{\headrulewidth}{0pt} % customise the layout...
\lhead{}\chead{}\rhead{Méthode de Sensibilisation à la Sécurité Informatique -
Tiphaine Romand-Latapie}
\lfoot{\thepage}\cfoot{}\rfoot{\tiny{Copyright Orange \& Tiphaine Romand-LatapieTiphaine Romand-Latapie - 06/05/2015
Toute reproduction, représentation, utilisation ou modification est interdite
sans l'accord de l'auteur.
Orange – SA au capital de 10 595 541 532 e - 78 rue Olivier de Serres - 75505
Paris Cedex 15 - 380 129 866 RCS Paris
Tiphaine Romand-Lataoie : tiphaine.romand@orange.com}
}

%%% SECTION TITLE APPEARANCE
\usepackage{sectsty}
\allsectionsfont{\sffamily\mdseries\upshape} % (See the fntguide.pdf for font
%help)
% (This matches ConTeXt defaults)

%%% ToC (table of contents) APPEARANCE
\usepackage[nottoc,notlof,notlot]{tocbibind} % Put the bibliography in the ToC
\usepackage[titles,subfigure]{tocloft} % Alter the style of the Table of
%Contents
\renewcommand{\cftsecfont}{\rmfamily\mdseries\upshape}
\renewcommand{\cftsecpagefont}{\rmfamily\mdseries\upshape} % No bold!

%%% END Article customizations

%%% The "real" document content comes below...

\title{Méthode de Formation : Sensibilisation à la Sécurité Informatique pour un
Public Néophyte v0.2}
\author{Tiphaine Romand-Latapie} 
\date{06/05/2015} % Activate to display a given date or no date (if empty),
         % otherwise the current date is printed 

\begin{document}
\maketitle

\begin{abstract}
Ce document décrit une méthode de sensibilisation d'un public néophyte à la
sécurité informatique. Cette méthode est basée sur l'utilisation d'un jeu de
rôle, inventé par l'auteur. Le lecteur trouvera dans ce document les
informations lui permettant de réaliser lui-même cette formation (soumis à
autorisation de l'auteur).
\end{abstract}

\section{Copyright}
Ce document, la formation qu'il décrit et la méthodologie présentée sont
propriétés de leur auteur, Tiphaine Romand-Latapie. La formation a été réalisée
pour la première fois le 06 Mai 2015, à partir d'une idée originale de l'auteur.
Le document est protégé par un Copyright Orange \& Tiphaine Romand-Latapie.
Toute reproduction, représentation, utilisation ou modification est interdite
sans autorisation de l'auteur.
\\
Joindre l'auteur : 
\begin{verbatim}tiphaine.romand@orange.com\end{verbatim}
\begin{verbatim}tiphaineRL@gmail.com\end{verbatim}
\section{Introduction}
L'idée de cette formation est née de la nécessité de former un public
opérationnel néophyte aux enjeux de la sécurité informatique. Plutôt que de
rentrer dans un mécanisme de formation standard, basé sur la compréhension du
contexte technique (qu'est-ce qu'un mot de passe, comment fonctionne un
ordinateur etc.),
qui, selon mon expérience, a tendance à ennuyer ou effrayer un public non
informaticien, j'ai souhaité me concentrer sur les principes génériques de la
sécurité informatique:
\begin{itemize}
\item La problématique de « faire confiance » à une entité/personne ;
\item La notion de « défense en profondeur » ;
\item	Les motivations de l'attaquant ;
\item La démystification de l'attaquant (qui n'est pas forcément un « hacker de
génie ») ;
\item	Les notions de compromis entre contraintes opérationnelles et sécurité  ;
\item Les objectifs des équipes de sécurité (prévoir le comportement de
l'attaquant, le prévenir ou le détecter) .
\end{itemize}
L'idée sur laquelle est basée cette formation est la suivante : les macro
principes suivants sont les mêmes en sécurité physique et informatique :
\begin{itemize}
\item Nous sommes tous les jours confrontés à la notion de « faire confiance » à
quelqu'un
\item En sécurité physique, nous travaillons toujours dans le pire cas, nous
traitons de plus les cas où la mesure de sécurité préliminaire est
désactivée/inopérante
\item	L'attaquant est motivé par l'argent/l'idéologie etc.
\item	Les attaquants les plus répandus ne sont pas des génies du crime 
\end{itemize}
La sécurité physique impose elle aussi des contraintes (fermer la porte à clé,
port de badge obligatoire, contrôle avant de prendre l'avion, etc.)
Les objectifs sont ici communs : prévoir le comportement d'un attaquant, le
prévenir ou le détecter.

Or, les personnes néophytes sont beaucoup plus familières avec la sécurité
physique que la sécurité informatique, autant dans leur vie professionnelle que
personnelle. Tout le monde ferme sa porte à clé avant de sortir de chez soi,
personne ne laisse rentrer n'importe qui chez lui, tout le monde a déjà eu à
faire à des contrôles de sécurité, etc.

L'idée au cœur de la formation est donc de faire prendre conscience au public
formé qu'il dispose déjà des bons réflexes et raisonnements, et de lui apprendre
à les appliquer à la sécurité informatique, tout en dédramatisant cette
dernière.


\section{Le jeu de rôle}
La formation est construite autour d'un jeu de rôle basé sur l'attaque et la
défense d'un bâtiment.

\subsection{Règles du Jeu}
Le jeu se déroule avec un Animateur, appelé aussi « Maître du Jeu », une équipe
d'attaquants et une équipe de défenseurs.

\subsubsection{Description générale :}
\begin{itemize} 
\item Un immeuble de bureau dans une zone dense, avec parking souterrain et
hélipad (piste d'atterrissage d'hélicoptère sur le toit). Un objet (tenant dans
un sac à dos) de grande valeur, utilisé par des employés pendant la journée est
stocké quelque part dans l'immeuble.
    \item Au début du jeu l'immeuble n'est pas sécurisé. 
\item Les attaquants proposent une attaque, les défenseurs une contre-mesure, et
on recommence.
\end{itemize}

\subsubsection{Règles et objectifs de l'équipe d'attaquants :}
\begin{itemize}
    \item Objectif : voler l'objet dans l'immeuble sans finir en prison
\item Règles : budget illimité – nombre d'attaquants humains inférieur à dix -
respecter les lois de la physique
\end{itemize}

\subsubsection{Règles et objectifs de l'équipe de défenseurs :}
\begin{itemize}
\item Objectif : empêcher le vol de l'objet ou récupérer de quoi faire arrêter
les voleurs
\item Règles : Budget illimité – personnel illimité – respecter les lois
françaises et les lois de la physique – des gens doivent pouvoir travailler dans
l'immeuble pendant la journée
\end{itemize}

\subsubsection{Fin d'un scenario :}
\begin{itemize}
\item Je conseille de terminer un échange (appelé « scenario ») lorsque les
équipes arrivent à un point de blocage (tout le monde est mort, l'objet est
détruit, les policiers sont là …)
\item Il est alors possible de passer à une nouvelle tentative (reprise à zéro)
des attaquants. Dans ce cas : les défenseurs conservent toutes leurs mesures de
protection.
\item Dans le cas où les joueurs le souhaitent, ou si le maître du jeu souhaite
relancer le jeu, il est également possible d'inverser les équipes (les
attaquants deviennent défenseurs etc.)
\end{itemize}

\subsubsection{Fin du jeu :}
\begin{itemize}
    \item Il n'y a pas de gagnant ni de perdant ! 
\item Il est conseillé de faire plusieurs échanges ou « scenarii » dans une
seule session, en ce cas, la fin du jeu est laissée à la discrétion de
l'animateur (nous conseillons un jeu d'une durée de 40 à 60 minutes pour 6
personnes).
\item Le jeu est suivi d'un débriefing par le formateur, permettant de mettre en
exergue les notions souhaitées (voir la section consacrée au débriefing).
\end{itemize}
	
\subsection{Concept des règles}
\subsubsection{Environnement}
L'environnement du jeu (immeuble de bureau, zone dense, etc.) a été choisi pour
maximiser le côté ludique et faciliter l'application à la formation :
\begin{itemize}
\item L'immeuble de bureau utilisable pendant la journée permet de travailler
sur les compromis contrainte/sécurité et offre un environnement familier des
joueurs.
	\begin{itemize}	
\item Ne pas hésiter à personnaliser les détails du scénario avec
l'environnement professionnel des joueurs : bâtiment de la société, objet de
valeur correspondant à un produit phare de l'entreprise, etc. Cela permet une
immersion et implication des joueurs plus rapide (sans compter le plaisir à
virtuellement perturber le quotidien professionnel).
	\end{itemize}
\item Le choix de la zone dense, de l'hélipad et du parking souterrain permet de
renforcer le côté ludique (les attaquants peuvent imaginer creuser un tunnel, se
poser en hélicoptère, sauter d'un immeuble à l'autre etc.) et de guider un peu
les joueurs. Cela permet également de forcer la diversification des scenarii
d'intrusion.
\item Le choix de ne pas plus préciser l'environnement permet de laisser libre
cours à l'imagination des joueurs, et de simplifier les règles.
\item L'utilisation de l'objet pendant la journée permet d'éviter des mesures
non constructives pour le jeu, du type « je coule l'objet dans un bloc de béton
».
\item L'emplacement de l'objet est laissé libre, il peut évoluer au cours du
jeu.
\item	Le fait de commencer sans sécurité est important : 
	\begin{itemize}	
\item Il permet de travailler sur l'empilement des mesures de sécurité et sur le
principe selon lequel l'attaquant passe toujours par le point de moindre
résistance ;
\item Les attaquants partent régulièrement du principe que l'immeuble « sans
sécurité » comporte quand même des caméras de surveillance, des portes qui
ferment à clé etc. Dans ce cas il n'est pas nécessaire de recadrer le jeu. En
revanche, il est intéressant de faire réfléchir les joueurs sur ce sujet au
cours du débriefing.
	\end{itemize}
\item	Les échanges rapides permettent un jeu vivant et ludique.
\end{itemize}

\subsubsection{Les règles et objectifs des attaquants :}
\begin{itemize}
\item Un objectif simple, renvoyant aux films à gros budgets, facile à traduire
en objectifs de sécurité informatique (entrée-sortie sans laisser de trace) ;
\item Le budget illimité simplifie le jeu, tout en conservant la possibilité de
discuter des aspects financiers lors du débriefing ;
\item Le petit nombre de personnes physiques auquel a droit l'attaquant permet
d'éviter les situations irréalistes du type « une armée de 300 personnes fait le
siège du bâtiment » ;
\item Le respect des lois de la physique permet encore une fois d'éviter des
situations irréalistes et contraires à l'esprit du jeu (pas de
téléportation/magie etc.).
\end{itemize}

\subsubsection{Les règles et objectifs des défenseurs :}
\begin{itemize}
\item Un objectif simple, renvoyant aux films à gros budgets, facile à traduire
en objectifs de sécurité informatique (contrôle des entrées/sorties, ralentir
l'attaquant etc.) ;
\item Le budget illimité simplifie le jeu, tout en conservant la possibilité de
discuter des aspects financiers lors du débriefing ;
\item Le personnel illimité permet de compenser un peu le besoin de respecter
les lois françaises, tout en permettant de faire un lien entre les mesures de
protections parfois très chères mais inefficaces ;
\item Le respect des lois de la physique permet encore une fois d'éviter des
situations irréalistes et contraires à l'esprit du jeu (pas de
téléportation/magie etc.) ;
\item Le respect des lois du pays renvoie aux contraintes des ingénieurs en
sécurité informatique, qui sont eux-mêmes limités par les lois du pays dans
lequel ils pratiquent.
\end{itemize}

\subsubsection{Qui perd gagne}
Il n'y a pas de perdant ou de gagnant, même si les équipes de joueurs ont
tendance à en vouloir un. Des règles permettant de désigner des
gagnants/perdants complexifieraient inutilement le jeu. Le but des règles est
globalement de favoriser les échanges ludiques entre joueurs, tout en faisant
ressortir les notions dont le formateur a besoin pour que la formation atteigne
son but.

\subsection{Animation du jeu}
Le jeu est animé par le(s) formateur(s), nommé ``maître du jeu'' dans la suite.
Il est important de ne pas faire de trop grosses équipes. Je conseille 2 à 3
défenseurs et 2 à 3 attaquants. Au-delà de ce nombre, il devient très difficile
pour le formateur de suivre les échanges et de les recadrer.

Le formateur commence par expliquer l'objectif du jeu et ses règles : \\

Objectif du jeu : faire prendre conscience aux élèves du travail des équipes de
sécurité et du fait qu'ils possèdent déjà les bons réflexes : la formation doit
leur donner les clés pour les appliquer à l'informatique.

\subsubsection{Règles du jeu} 
Il est important de bien insister sur l'aspect « physique » du jeu. Dans
certains groupes, les élèves, conscients d'être dans une formation à la sécurité
informatique, cherchent immédiatement à « pirater » quelque chose. Il faut
également bien préciser que pour les deux camps, l'objectif est double
(empêcher/réaliser le vol ou le détecter), ceci pour permettre de faire émerger
les notions d'usurpation d'identité, de traces etc. Enfin, il ne faut pas
hésiter à insister sur les aspects juridiques du pays: les attaquants ont tous
les droits, mais pas les défenseurs.

\subsubsection{Déroulement du jeu}
Dès le lancement du jeu, le formateur doit noter sur un support visible par les
participants les différents échanges (cf. les exemples fournis dans ce
document). En tant que Maître du Jeu, il est responsable du bon respect des
règles, et peut limiter l'un ou l'autre des camps.

Il doit forcer les attaquants/défenseurs à détailler leurs actions dès que
nécessaire :
\begin{itemize}
\item si quelque chose est verrouillé, on doit savoir par quel type de mesure
(empreinte rétiniennes, empreintes digitales, badges, code, clé), et qui possède
l'élément permettant de déverrouiller ;
\item en cas par exemple de passage sur générateur de secours, il faut préciser
ce que celui-ci alimente et le maître du jeu peut choisir de limiter le temps de
fonctionnement du générateur de secours. Par exemple : un générateur de secours
au fioul ne peut pas alimenter un système de sécurité complet d'un immeuble de
bureau plus de quelques heures ;
\item en cas de caméra de surveillance, il faut préciser si elles sont
surveillées en temps réel, par qui et combien sont-ils etc.
\end{itemize}
La nécessité de préciser telle ou telle action est décidée par le maître du jeu,
en fonction des enseignements qu'il souhaite tirer du jeu pendant le débriefing.
Nous conseillons toutefois vivement de faire préciser les cas cités ci-dessus,
ainsi que l'emplacement d'un PC footnote{Poste de Commandement} sécurité par
exemple.

Tout ce qui n'est pas explicitement dit par les défenseurs ou les attaquants
peut être interprété/détourné par le camp adverse. Si les défenseurs n'indiquent
jamais avoir fermé les fenêtres, les attaquants peuvent considérer qu'elles sont
ouvertes. Si les attaquants n'ont pas dit qu'ils étaient masqués, il faut
considérer que les caméras de surveillance filment leurs visages etc.

Le maître du jeu peut orienter l'un ou l'autre des camps s'il trouve que le jeu
ne va pas dans la bonne direction,
ou si les échanges sont laborieux. Il peut rappeler par exemple les règles au
moment opportun, comme dire à une équipe d'attaquants timides « je vous rappelle
que vous n'avez pas à respecter les lois françaises, vous pouvez faire exploser
la porte/tuer le garde ». L'objectif du maître du jeu est de faire ressortir
dans le jeu (ou de repérer) les éléments lui permettant d'illustrer, lors du
débriefing, les principes de base de la sécurité informatique.

Aucune comparaison/lien avec la sécurité informatique ne doit être réalisé
pendant le jeu de rôle. Les liens sont effectués lors du débriefing.

\subsubsection{Fin du jeu}
Il est conseillé de terminer le scenario en cours dans les cas suivants : 
\begin{itemize}
\item Les attaquants s'entêtent dans une direction alors que d'autres
possibilités n'ont pas été explorées ;
\item	Le scenario en cours devient trop complexe ; 
\item	Le scenario en cours devient irréaliste ; 
\item	Le formateur souhaite inverser les équipes ; 
\item Les joueurs perdent en motivation (il est alors possible d'arrêter le jeu
ou d'inverser les équipes)
\item	Le formateur a déjà toute la matière dont il a besoin pour son débriefing.\end{itemize}

\subsection{Exemple d'échange/scenario}

Cet exemple a été observé au cours d'une des formations, à ce moment, le jeu
durait déjà depuis 10 minutes.


\begin{longtable}{|p{4cm}|p{4cm}|p{4cm}|}
\caption[Exemple de scenario]{Exemple de scenario}\\

\hline
Attaquants & Défenseurs & Maître du jeu \\
\hline
\endhead

\hline \endfoot
          
Corruption de sous\-traitants pour qu'ils réalisent eux-mêmes le vol & & \\ 
&L'objet en utilisation reste en visibilité permanente de son utilisateur. Dès
qu'il n'est plus utilisé, il est rangé dans un coffre-fort fermant à clé. Trois
personnes ont chacune une clé: Le responsable du service de l'utilisateur de
l'objet, l'utilisateur de l'objet et le directeur de la sécurité. On trace les
actions des responsables de ces trois clés en permanence. & Qui possède la clé
du coffre ? \\
Récupération du nom du directeur de la sécurité, phase d'observation pour
connaître son emploi du temps. Vol avec violence permettant de récupérer la clé
et de la donner au sous-traitant & & \\
& Le coffre n'est pas en évidence & Mesure non effective : Le personnel
d'entretien peut le trouver facilement.\\
& Vidéo Surveillance multi-écran. Un gardien 24/24 pour surveiller les écrans et
un enregistrement. Un gardien est également présent à l'accueil & Attention,
trop de caméras implique difficulté/impossibilité pour un humain de les
surveiller en temps réel \\
Une femme de ménage distrait le gardien de la vidéo surveillance pendant que
l'autre commet le vol & & \\
& Formation accrue des gardiens (par la police, les forces spéciales etc.),
enquête de moralité des sous-traitants & Il sera toujours possible de trouver un
sous-traitant « faible », le gardien pourrait être malade, avoir besoin d'aller
aux toilettes etc. Mais les attaquants ont « perdu » ! Les enregistrements vidéo
sont revus et contiennent le visage de la femme de ménage \\
La femme de ménage se déguise dans les toilettes/La personne distrayant le
gardien utilise un appareil permettant de détruire à distance les données sur
disque dur (aimant) & & \\
& Vidéo-Surveillance dans le couloir devant les toilettes et salle serveur
protégée (au centre de l'immeuble, avec cage de faraday). & La caméra a été mise
devant les toilettes suite à un recadrage du Maître du Jeu, la loi interdit de
mettre des caméras dans les toilettes ! \\
Débrancher la caméra & & \\ 
& Alarme sonore et visuelle se déclenchant dans la loge du gardien en cas de
dysfonctionnement/débranchement de caméra & Le Maître du Jeu déclare la fin du
scenario, pour forcer l'équipe d'attaquant à passer à autre chose.\\
\end{longtable}

\section{Le débriefing du jeu de rôle}
\subsection{Apprentissage des bons réflexes « communs » } 
Comme indiqué en introduction de ce document, les personnes néophytes 
ont déjà de bons réflexes, qui peuvent être appliqués aussi bien en sécurité
physique qu'en sécurité informatique. Je conseille de présenter ces reflexes en
sortie de jeu de rôle, en faisant le lien avec les scenarii apparus durant le
jeu. Voici une liste non-exhaustive de bons réflexes à mettre en avant par le
formateur :

\begin{itemize}
\item	Ne pas « faire confiance » par défaut ;
\item On vérifie les identités ;
\item On ne donne pas la clé de sa maison/son code d'alarme/mot de passe à
n'importe qui
\item	Cas des services d'urgence : donnez-vous votre clé « au cas où »?
\item	Pourquoi donnerait-on son mot de passe au SAV ?
\item On appelle la police/les services de sécurité en cas de suspicion
d'activité malveillante
\item	On se pose les questions : 
\item	« Quelqu'un a-t-il intérêt à attaquer mon bâtiment ? », « À quel point ? »
\item	Cette information/clé/badge serait-il utile à quelqu'un ?
\item	Que se passe-t-il en cas de dysfonctionnement ?
\end{itemize}

\subsection{Grille de lecture des scenarii}

Les éléments utilisés lors du jeu de rôle par les différents participants
ont une correspondance facile avec la sécurité informatique. L'idée du
débriefing est de réaliser ce lien entre le jeu et les aspects de la sécurité
informatique sur lesquelles le formateur souhaite insister. La figure 1 présente 
une grille de lecture (non exhaustive) des éléments généralement utilisés par les joueurs et
de ce qu'ils peuvent représenter dans le milieu de la sécurité informatique.

\begin{figure}
    \begin{tabularx}{\textwidth}{|X|X|}
    \hline
Sécurité Physique & Sécurité Informatique \\ \hline
Clé/Badge & Mot de passe/carte à puce \\
Coffre/Porte blindée & Mesure technique de sécurité \\
Vidéo Surveillance & Supervision/logs/anti-virus \\
Destruction des enregistrements de vidéo surveillance & Destruction/Altération
des logs \\
Coupure d'électricité/incendie	                        & Denis de Service \\    
Gardes/gardiens/personnel	                            & Opérationnels \\
Déguisement/fausse carte d'identité & Usurpation d'adresse IP, usurpation
d'identité \\
Observation, récupérer le nom d'un chef, une information … & Social Engineering
\\
Procédure d'urgence, générateur de secours, etc. & Résistance aux pannes,
défense en profondeur, procédure SAV \\
    Carte d'identité 	                                    & Certificats \\
Utilisation par les attaquants d'une technologie spécifique (brouilleur,
explosifs, drones etc.) & Utilisation d'exploit/de plateforme écrits par
d'autres \\ \hline
    \end{tabularx}
    \caption{Grille de Lecture}
\end{figure}

\subsection{Points Communs et Divergences}

Les points communs entre les sécurités physique et informatique ont déjà été
présentés plusieurs fois au cours de ce document, nous les reprenons maintenant
et indiquons les exemples types illustrant ces principes et apparaissant dans le
jeu de rôle.

\subsubsection{La problématique de « faire confiance » à une entité/personne }
Très vite dans le jeu, les participants sont confrontés à la notion de contrôle
d'accès. Vous verrez assez vite apparaître des notions de badge/vérification de
cartes d'identité à l'accueil, ou d'attaquant se déguisant ou mentant pour
accéder à l'immeuble. Il est important d'utiliser ces points pour faire
réfléchir les participants à la notion de confiance, d'identité et
d'authentification. L'utilisation d'une fausse carte d'identité par l'attaquant
est par exemple intéressante : qu'est ce qui nous permet de croire quelqu'un
quand il décline son identité ? Cette notion est centrale dans tout système de
sécurité. Le formateur peut également profiter de cette discussion pour parler
des différentes possibilités :
\begin{itemize}
\item	Biométrie
\item	PIN ou mot de passe
\item	Clé (qui peut être volée, perdue, copiée etc.)
\item Carte d'identité, qui renvoie à la notion de faire confiance à une tierce
partie (l'état dans la sécurité physique, une autorité de confiance dans la
sécurité informatique)
\end{itemize}

Enfin, dans la plupart des sessions effectuées, les attaquants utilisent assez
vite des mensonges/usurpation d'identité pour contourner des mesures de
sécurité. Par exemple, dans une session, les attaquants récupéraient le nom d'un
manager haut placé, et insistaient sur une livraison urgente à ce dirigeant à
l'accueil. Ce genre de scenario est très utile pour illustrer le concept de
social engineering. C'est enfin l'occasion de faire réfléchir les élèves sur la
maxime « l'élément le plus faible est l'humain ».

\subsubsection{La notion de « défense en profondeur »}
Le principe de défense en profondeur, qui consiste à empiler les couches de
sécurité et à traiter les cas où une couche est défaillante, apparait facilement
lors du jeu de rôle. Par exemple, de façon systématique, les élèves proposent un
contrôle d'accès à l'entrée de l'immeuble, puis un contrôle d'accès accru à la
pièce dans laquelle est stocké l'objet. Ils peuvent même rajouter un contrôle
supplémentaire autour de l'objet dans cette pièce.

Le formateur se doit de mettre en avant ce comportement, et de faire remarquer
aux élèves qu'il en est de même dans la sécurité informatique. C'est le moment
de discuter avec eux des mesures de sécurité multiples, et de faire prendre
conscience de leur intérêt. Nous entendons souvent, en tant qu'ingénieur en
sécurité informatique des phrases du type « mais c'est dans le LAN, nous ne
risquons rien », « mais là nous avons déjà tapé un mot de passe une heure avant,
pourquoi un autre ? » Etc.

La multiplication du type de technologie (clé physique, badge, biométrie etc.)
est aussi une façon de faire réfléchir les élèves sur les règles d'hygiène
informatique (une clé/mot de passe par usage, etc.).
Enfin, les différentes tentatives des attaquants permettent d'illustrer très
bien la règle d'or selon laquelle le niveau de sécurité d'un système dépend du
niveau de sécurité de son composant le plus faible.

\subsubsection{Les motivations de l'attaquant}
Les différents scenarii permettent au formateur d'illustrer la notion importante
de motivation de l'attaquant (et du défenseur). Lorsque nous arrivons à des
scenarii qui représentent plusieurs millions de d'euros, et des mois de
préparation, la question se pose : l'objet en vaut-il la peine ? La même
question peut être posée pour les défenseurs.

C'est également le moment de discuter du niveau de sécurité par rapport au
niveau de l'attaquant, et de faire réfléchir les élèves aux questions au cœur de
tout système de sécurité : que protège-t-on et contre qui ?

\subsubsection{La démystification de l'attaquant (qui n'est pas un « hacker de
génie »)}
Une des notions les plus mal perçues par les néophytes est la diversité des
profils d'attaquants informatique. L'imaginaire collectif dépeint une image de «
génie » au fond d'une cave, or, comme en sécurité physique, il y a plusieurs
types d'attaquants : si la porte ne ferme pas à clé, n'importe quel délinquant
peut entrer dans l'immeuble. Lorsque le scenario devient complexe, nous sommes
face à des attaquants extrêmement motivés ciblant un objectif bien défini.

La notion d'économie souterraine est également mal comprise : 
\begin{itemize}
\item En sécurité physique les objets sont revendus ou « commandés » avant le
vol. La même chose existe en sécurité informatique et les élèves doivent en
prendre conscience ;
\item De même qu'un attaquant « physique » va acheter des outils lui permettant
de réussir son attaque (explosifs, brouilleurs radio, fausses carte d'identités
etc.), l'attaquant informatique fait de même. Ce qui veut dire qu'il y a une
économie liée à la découverte de ces outils (failles, exploit etc.) et à leur
revente. C'est le moment de faire réfléchir les élèves sur les différents
profils. N'importe qui peut appuyer sur le bouton d'un brouilleur radio, il faut
en revanche des compétences techniques poussées pour le concevoir.
\end{itemize}

\subsubsection{Les notions de compromis entre contraintes opérationnelles et
sécurité }
Pour cette notion, le formateur doit se concentrer sur les mesures mises en
œuvre par les défenseurs, et des contraintes qu'elles impliquent pour les
employés de la société ou la société elle-même. Le lien est alors assez facile à
faire avec les contraintes de sécurité informatique.

Une partie intéressante à travailler est la présence dans les scenarii d'équipe
de secours (police, armée, pompiers…) légitimes ou non. Demander aux élèves :
donnent-ils accès à tout l'immeuble de façon systématique à toutes les équipes
de secours « au cas où » ? Vérifient-ils la légitimité des services de secours ?
Dans une des sessions, les attaquants s'étaient faits passer pour une équipe
médicale d'évacuation par hélicoptère pour extraire l'objet volé. C'est le
moment de discuter des accès SAV/service informatique etc. Des stockages de mot
de passe en clair « au cas où le client l'oublie etc. ».

\subsubsection{Les objectifs des équipes de sécurité (prévoir le comportement de
l'attaquant, le prévenir ou le détecter)}
Dans le jeu de rôle, le travail des défenseurs est facilité par rapport aux
conditions réelles : les attaquants annoncent leur intention et leur objectif
est connu. Le formateur peut profiter du débriefing pour mettre le doigt sur la
difficulté des équipes de sécurité : elles doivent imaginer le comportement des
attaquants et évaluer leurs possibles motivations.
Le formateur peut également faire réfléchir sur les outils de supervisions de
traces. Lorsqu'il n'est pas possible de prévenir le comportement d'un attaquant,
il faut au minimum le détecter.

\subsubsection{Divergences}

Toute la sécurité physique n'est, bien sûr, pas transposable en sécurité
informatique (et vice versa). Mais les différences, aussi essentielles
soient-elles, sont finalement peu nombreuses sur le fond :
\begin{itemize}
\item	Le facteur temps diffère énormément : 
\begin{itemize}
\item Par exemple : tester un mot de passe est bien plus rapide que tester une
clé sur une serrure physique
\end{itemize}
\item Le facteur géographique n'existe pratiquement plus :
	\begin{itemize}
\item L'attaquant n'a aucun besoin d'être physiquement présent pour mener
l'attaque. La distance géographique n'a pas d'importance.
	\item	Il y a des exceptions à cette règle : 
		\begin{itemize}
\item Les lois qui s'appliquent sont liées à la localisation physique des
données volées ou altérées ;
\item Lors d'attaques de type signaux compromettants, radio ou hardware, la
distance géographique peut redevenir un élément déterminant ;
		\end{itemize}
	\end{itemize}
\item Ces deux changements d'échelle rendent les attaques de masse peu coûteuses
et accessibles à tous ;
\item Les traces exactes et facilement récupérables concernent les machines et
plus difficilement les personnes physiques
	\begin{itemize}
	\item	Il peut être très difficile de remonter au coupable ;
\item L'attaquant peut se cacher derrière des machines de tierces parties non
concernées ;
	\end{itemize}
\item	Le vol est pratiquement impossible à détecter (copie informatique)
	\begin{itemize}
\item Les traces du vol peuvent être retrouvées si le système est correctement
configuré ;
	\end{itemize}
\item Encore trop souvent, aucune sécurité basique n'est mise en place en
informatique, alors que dans le monde physique, les gens mettent en œuvre au
minimum une porte qui ferme à clé.
\end{itemize}

\section {Annexe : exemple d'une session}
Cette session a été réalisée avec cinq personnes (trois défenseurs, deux
attaquants), elle a duré environ 50 minutes.



\begin{longtable}{|p{3cm}|p{3cm}|p{3cm}|p{3cm}|}
    
\caption[Cas réel]{Exemple d'une session complète à 5 joueurs d'une durée de 50
minutes environ (hors débriefing)}\\
    
\hline 
Attaquants	& Défenseurs	& Commentaires	 & Parallèle sécurité informatique \\
\hline 
\endhead 
\hline \endfoot
Ouvre la porte, va chercher l'objet, ressort avec l'objet & & & Vol de données
non protégées \\
& La porte est sécurisée par badgeuse et est fermée complètement après 20h. Si
tentative d'effraction, alarme relié au commissariat & & Protection par mot de
passe, gestion des droits utilisateurs, supervision \\
Une femme séduit un employé et lui explique qu'elle a oublié son badge,
l'employé la fait entrer (vol puis sortie) & & & Social engineering \\
& L'objet est enfermé dans un coffre-fort à code. Seul le responsable du site
dispose du code (il faut l'appeler pour utiliser l'objet). Le port du badge
apparent est obligatoire dans l'entreprise, des agents de sécurité vérifient
l'application de la consigne. Les salariés sont de plus sensibilisés sur le
danger de laisser entrer un inconnu. & On remarque que la mesure est très
contraignante pour l'entreprise (une seule personne pour l'accès à l'objet qui
doit être utilisable) & Protection par mot de passe. Non partage des mots de
passes. Logs et supervision. Sensibilisation \\
Déguisement en personnel d'entretien, entre avec un badge volé et un chariot qui
contient un chalumeau. Ouvre le coffre au chalumeau, récupère l'objet, le met
dans le chariot et ressort. & & & Usurpation d'identité, attaque par force brute
\\
& Il y a un détecteur de fumée dans la pièce. L'accès à la pièce est protégé par
un système de reconnaissance rétinien. & & Détection d'attaque. Biométrie \\
L'attaquant se pose en hélicoptère sur le toit de l'immeuble, puis utilise les
conduits de climatisation jusqu'à la pièce. Descente "à la mission impossible"
et vol du coffre. Sortie de l'immeuble, ouverture du coffre etc. & & & Attaque
"Hors ligne", vol puis tentative de contournement de la mesure de sécurité. \\
& Le coffre est scellé dans le mur, il est de plus électrifié si le système de
reconnaissance rétinien n'a pas validé l'entrée dans la pièce. & & Interdiction
d'attaque "hors ligne". Interdiction de toute action sans mot de passe \\
Coupure d'électricité & & & Denis de service/panne du système de sécurité \\
& Groupe électrogène de secours maintenant les fonctions de sécurité de la pièce
& & Système de secours \\
L'attaquant fait chanter un employé ayant les accès à la pièce en prenant en
otage sa famille. L'employé réalise le vol & & & Social engineering \\
&Des caméras sont présentes devant et dans la pièce, elles sont surveillées
24/24h par du personnel au PC Sécurité qui se trouve à l'extérieur du bâtiment.
& & Logs et supervision sur des serveurs dédiés \\
L'attaquant fait exploser le PC sécurité & & & Destruction, modification des
logs \\
& En cas d'explosion ou de coupure du PC sécurité, des équipes sont envoyées au
PC sécurité et dans l'immeuble. Une alarme se déclenche en cas de coupure du
lien de communication entre PC sécurité et batiment principal & & Protection des
logs, défense en profondeur (mécanisme de secours du mécanisme de secours) \\
Piratage des caméras de sécurité pour couper le flux vidéo & & & Attaque du
système écrasant les logs \\
&Un capteur de mouvement est placé sur l'objet, en cas de mouvement, l'objet
explose. & Refusé : l'objet doit être utilisable pendant la journée. non
conforme avec la loi française. & En informatique, on appelle cette technique
l'effacement d'urgence. En cas d'attaque détectée, les données sensibles sont
effacées. Très contraignant \\
Intrusion en passant par les champs morts de la caméra et vol d'un badge pour
entrée & & & \\
& Tous les angles de caméras sont couverts et la surveillance se fait par une
personne par caméra& Mesure très chère en personnel & Augmentation de la
supervision et du personnel opérationnel \\
    Cache caméra montrant une photo de l'emplacement	 	& & & \\ 	 
& Un garde est présent à l'accueil et contrôle les entrées, des rondes sont de
plus effectuées avec chiens de garde & & \\
L'attaquant attaque les gardes et nourri les chiens pour les distraire & & & \\
&Toujours un contrôle rétinien, un ajout de contrôle d'empreinte pour accès à la
salle. & & Contrôle d'accès biométrique \\
    Salarié infiltré (taupe) qui réalise le vol	 	& & & \\ 	 
&Fouille systématique des employés du site & Mesure très contraignante
(plusieurs secondes par employés, en horaire de pointe …) & Contrôle permanent
de l'ensemble du contenu des ordinateurs des employés (interdit) \\
Utilisation d'un drone pour faire sortir l'objet & & & Exfiltration de données
\\
&Fouille à l'entrée/sortie de la pièce. & Mesure très contraignante (plusieurs
secondes par employé, en horaire de pointe …) & \\
    Meurtre du garde surveillant la pièce & && \\	 	 	 
&Un antivol permet de détecter que l'objet est sortie du bâtiment, le site est
verrouillé & & Watermarking des données (moins efficace) \\
Déclenchement d'un incendie pour ouverture automatique des portes d'évacuation &
& & Attaque des procédures de secours/d'urgence \\
        & Géolocalisation de l'objet 	 & & \\	 
    Utilisation de Papier aluminium pour empêcher la détection 	  & & & \\	 	 
        & Intervention de l'armée pour abattre le drone 	 & & \\	 
    Plusieurs centaines de drones font diversion	 	& & & \\ 	 
        &Brouilleur radio pour empêcher le pilotage des drones	 	& & \\ 
    Drones autoguidés, préprogrammés	 & & & \\	 	 
        & Brouillage des signaux GPS	&  & \\ 	 
Passage par le sous-sol pendant que le drone sort un leurre de l'objet, sortie à
3 motos, seule l'une à l'objet & & & Diversion/surcharge de la supervision \\
&Des clous ont été mis sur a route lors du déclenchement de l'alarme. Porte
blindée en sortie de parking & & Toutes les notions attendues ont été exprimées,
de plus, le scenario devient trop complexe. Arrêt du jeu. \\

\end{longtable}

\end{document}
